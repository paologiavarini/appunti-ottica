\documentclass{article}
\usepackage[a4paper, total={15cm, 25cm}]{geometry}
\usepackage[utf8]{inputenc}
\usepackage{amsmath}

\title{Ottica Fisica}
\author{}
\date{}

\begin{document}
\maketitle




\section*{Ottica Geometrica}
\subsubsection*{Principio di Fermat}
Il raggio è la traiettoria che la luce percorre con minor tempo di percorrenza.\\
Legge di trasformazione dei raggi:
\begin{equation}	\label{eq:legge_di_trasf_dei_raggi}
n_1 \sin\theta_i - n_2 \sin\theta_t = 0
\end{equation}
La quale è simile alla legge di Snell ma, per come è stata ricavata, la \eqref{eq:legge_di_trasf_dei_raggi} non si limita ad un'onda incidente su una superficie piatta ma include qualsiasi geometria dell'interfaccia.

\subsubsection*{Immagine reale e immagine virtuale}
Si ha un'immagine reale quando i raggi che giungono al sensore provengono da un punto che coincide con la posizione dell'oggetto stesso. Nel caso in cui il punto dove convergono i raggi non sia quello dove l'oggetto è posizionato allora si parla di immagine virtuale.

\subsection*{Superfici cartesiane}
$\cdots$

\subsection*{Regime parassiale}
Il regime parassiale consiste nel considerare solo raggi incidenti il sistema ottico con angoli molto piccoli rispetto all'asse di quest'ultimo.\\
Con questa approssimazione possiamo associare una relazione lineare tra la posizione e l'angolo di entrata e uscita del raggio nel e dal sistema ottico.\\
Questa relazione è scrivibile sotto forma di matrice del tipo:
\[
\begin{bmatrix}
y_1\\
\alpha_1
\end{bmatrix}
=
\begin{bmatrix}
A	&	B\\
C	&	D
\end{bmatrix}
\begin{bmatrix}
y_0\\
\alpha_0
\end{bmatrix}
\]

\subsubsection*{Matrice di propagazione libera}
Nel caso in cui il sistema ottico non contenga nessun elemento ottico, allora si ha una propagazione libera del raggio su un percorso $L$.
\[
\begin{bmatrix}
y_1\\
\alpha_1
\end{bmatrix}
=
\begin{bmatrix}
1	&	L\\
0	&	1
\end{bmatrix}
\begin{bmatrix}
y_0\\
\alpha_0
\end{bmatrix}
\]

\subsubsection*{Matrice per superfici sferiche}
\[
\begin{bmatrix}
y_1\\
\alpha_1
\end{bmatrix}
=
\begin{bmatrix}
1	&	0\\
\frac{1}{R}(\frac{n}{n'}-1)	&	\frac{n}{n'}
\end{bmatrix}
\begin{bmatrix}
y_0\\
\alpha_0
\end{bmatrix}
\]
Questa relazione vale per qualsiasi superfice sferica di raggio $R$ sia essa concava o caonvessa. Se la superficie è convessa (se C si trova a destra del vertice V) R è assunto positivo altrimenti negativo.

\subsubsection*{Generalizzazione dei sistemi ottici}
Il concetto di sistema ottico è stato introdotto per semplificare l'analisi di sistemi con elementi ottici numerosi. Associando ad ogni elemento ottico la sua matrice $M_i$ è possibile ricavare la matrice di trasferimento $M$ totale, detta anche diottro sferico, che relaziona entrata e uscita dei raggi da tutti i sistemi ottici come se fosse uno solo.
\[
\begin{bmatrix}
y_N\\
\alpha_N
\end{bmatrix}
=
M_N M_{N-1} \cdots M_2 M_1
\begin{bmatrix}
y_0\\
\alpha_0
\end{bmatrix}
\]
Osservazione:
Da notare che la moltiplicazione delle matrici segue il verso opposto (da destra a sinistra) del percorso del raggio (da sinistra a destra).\\
Osservazione:
Durante la risoluzione degli esercizi può essere opportuno verificare il calcolo della matrice $M$ calcolandone il determinante. Infatti deve essere:
\begin{equation*}
\det M = \frac{n_0}{n_f}
\end{equation*}
Con $n_0$ indice di rifrazione del mezzo da cui parte il raggio e $n_f$ indice di rifrazione del mezzo in cui arriva il raggio.

\subsubsection*{Proprietà della matrice di trasferimento}
$D = 0$\\
$A = 0$\\
$B = 0$\\
\centerline{$A = \frac{y_f}{y_0}$ \qquad ingrandimento del sistema ottico}
$C = 0$\\
\centerline{$D = \frac{\alpha_f}{\alpha_0}$	\qquad ingrandimento angolare}


\section*{Ottica Diffrattiva}
\end{document}
