\documentclass{article}
\usepackage{mathtools}
\usepackage{amsmath}
\usepackage{esint}
\usepackage[charter]{mathdesign}
\begin{document}
\title{OTTICA DIFFRATTIVA}

Lavoriamo sempre con le equazioni di Maxwell facendo un analisi in frequenza.
\begin{equation}
\textbf{E}(\textbf{r}, t)= \textbf{E}( \textbf{r} ) e^{-i \omega t}
\end{equation}

\begin{equation}
\nabla ^{2}\textbf{E} (\textbf{r}) + (\frac{\omega}{c})^{2}\frac{2}{\varepsilon}\textbf{E} (\textbf{r})=0
\end{equation}

\textbf{Si potrebbe risolvere l'equazione di Helmotz con le appropiate condizioni al contorno, ma non è facile quindi si deve cercare una seconda via.}

Analizziamo ad esempio un fronte d'onda piana che si propaga attraverso un apertura:

%immagine%
 se usassimo ancora l'ottica geometrica troveremmo che passano solo i raggi centrali mrntre gli altri no.


Se l'apertura è molto grande rispetto alla lunghezza d'onda le deflesioni sono meno evidenti e il fronte d'onda è ancora praticamente piano. se però riduciamo l'apertura questa approssimazione non è più accettabile.

Andremo a studiare il sistema per aperture abbastanza piccole da avere diffrazione evidente, ma sufficentemente grande da non variare la polarizzazione del campo entrante lasciando l'equazione di Helmotz scalare(NON FARLA DIVENTARE VETTORIALE)

\section{OTTICA DIFFRATTIVA IN REGIME SCALARE}

ipotizziamo di essere nel vuoto 
\begin{equation}
\varepsilon = \varepsilon 	_{0}
\end{equation}
chiamo il campo elettrico U(P) anzichè \textbf{E} (\textbf{r}) 		(Metto in evidenza il fatto di essere scalare)

\begin{equation}
\nabla ^{2}U(P) + (\frac{\omega}{c})^{2}\frac{2}{\varepsilon _{0}}U(P)=0
\end{equation}

Nell'ottica diffrattiva studiamo tutto ciò che ha dimensione finita 


%IMMAGINI%

prendiamo un sistema con apertura di forma non definita attraverso la quale passa un onda polarizzata linearmente: come sarà il campo dopo l'apertura?

\textbf{DEvo risolvere l'equazione di Helmotz con le apposite condizioni al contorno. dovrò quindi usare le funzioni di Grenn} 

%cenni sulle funzioni di green%


a formulare la seguente equazione son kirchoff, Raleigh, Sommerfild

\begin{equation}
U(P_{0}) = \frac{1}{i\lambda} \iint_\sigma U(P_{1}) \frac{e^{i k r_{01}}}{r_{01}} cos(\textbf{n} , \textbf{ r})dS
\end{equation}

%r dovrebbe avere il pecide 01 ma caffacnuclo%
\begin{equation}
k=\frac{\omega}{c}  = \frac{2\pi}{\lambda}
\end{equation}

Esiste una strada alternativa che risulta solo dopo avere vissto la soluzione di RAleigh Sommerfild: \textbf{da un punto di vista matematico ci accorgiamo di avere in mano delle onde sferiche}

Si tratta di un potente artificio matematico che scompone il campo nella sovrapposizione di tante funzioni elementari.

Scriveremo il campo di partenza come somma di un set di onde semplici la cui sovrapposizione alla fine ci garantisce di trovare il campo complessivo

 


\end{document}
